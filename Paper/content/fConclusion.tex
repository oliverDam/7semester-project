The present paper had as porpuose to study the performance of linear regression methods for myoelectric prostheses control taking in acount the limb position effect. However, the performance of the online results were not siginificant different depending on the limb position as other studies had shown for classification control scheme. This findings agree with a recently published study by Hwang et al. \cite{Hwang2017}. In their study different limb position were tested using Root Mean Square (RMS) as feature to train the regressors.
During the results examination it was found that there is no correlation between the offline and online results. On the one hand the offline outcomes illustrates overfitting of the regression models. On the other hand the online test yielded robust control of the wrist movements performanced in the three different limb positions. This coud be owing to the subjects$'$ ability to compensate for a poorer fitting model when visual feedback is provided.
For this particular study two different features were extracted MAV and LogVar based on previous investigations. Even though LogVar had present linear properties, the online resutls showed no significant difference in performance compared to MAV. 
As a result of this study we can conclude that linear regression methods have a great potential to future research and development in myolectric prosthesis, due to tha fact that is possible to achieve simultaneous and proportional control necessary for daily life tasks. 
		
	

	% on the last page of the document manually. It shortens
	% the textheight of the last page by a suitable amount.
	% This command does not take effect until the next page
	% so it should come on the page before the last. Make
	% sure that you do not shorten the textheight too much.