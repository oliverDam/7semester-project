
	
	
%	\begin{table}[h]
%		\caption{An Example of a Table}
%		\label{table_example}
%		\begin{center}
%			\begin{tabular}{|c||c|}
%				\hline
%				One & Two\\
%				\hline
%				Three & Four\\
%				\hline
%			\end{tabular}
%		\end{center}
%	\end{table}
	


	
	
	%Figure Labels: Use 8 point Times New Roman for Figure labels. Use words rather than symbols or abbreviations when writing Figure axis labels to avoid confusing the reader. As an example, write the quantity ÒMagnetizationÓ, or ÒMagnetization, MÓ, not just ÒMÓ. If including units in the label, present them within parentheses. Do not label axes only with units. In the example, write ÒMagnetization (A/m)Ó or ÒMagnetization {A[m(1)]}Ó, not just ÒA/mÓ. Do not label axes with a ratio of quantities and units. For example, write ÒTemperature (K)Ó, not ÒTemperature/K.Ó