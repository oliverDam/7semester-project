%discussion points
This study presents the performance of linear regression methods using sEMG to minimize the limb position effect in proportional and simultaneous control of lower arm prosthetics. Along the process different aspects have been exposed that are important to mention.\\
%subdivide in sections??
Potential outliers were discoverd during this study. Even though the final results present that is possible to achieve simultaneous and proportional control using Myo armband, it had been shown that the sampling rate of this device could affect the acquisition of sEMG data. This could be due to the fact that sEMG signals  rate is between 10-500Hz, however the sampling rate of the Myoarmband is 200Hz, which casuses important information in raw data to be eliminated.
Based on prevoius studies two different features were extracted in order to tran the regressors MAV and LogVar.
A study by Hanhe \cite{hanhe2014} et al. showed that LogVar presented linear properties and outperformed the other different features extracted for that particular research. However the obtained results in our study illustrates that MAV performance outperformed the LogVar results. 
Futhermore online results were not siginificant different depending on the limb position as other studies had shown for classification control scheme. This findings agree with a recently published study by Hwang et al. \cite{Hwang2017}. In their study different limb position were tested using Root Mean Square (RMS) as feature to train the regressors.
During the results examination it was found that there is no correlation between the offline and online results. On the one hand the offline outcomes illustrates overfitting of the regression models. On the other hand the online test yielded robust control of the wrist movements performanced in the three different limb positions. This could be owing to the subjects$'$ ability to compensate for a poorer fitting model when visual feedback is provided.
It have been proved that is possible to achieve simultaneous and proportional control in myoelectric prosthesis using linear regression techniques. 
\begin{itemize}
	\item inclusion of more subjects
%	\item sample rate of the myo armband - exclusion of test subjects
	%\item even though LogVar shows linearity in a previous study, it does not perform better in linear regression than a presumably non-linear feature
	%\item Both features does not show significant difference in control in different limb positions
	\item Whatever the inclusion of IMU shows 
	
%	\item It is possible to use regression as control method to yield no significantly different performance in variations of limb positions
	\item Use other regression methods and other features to analyse if it will result in better performance
	\item why does the inclusion of IMU improve the performance
%	\item No connection between offline and online tests, which also is shown in previous studies
	\item reasonable control can be archived when donning/doffing(we train the regressors with data, take of the myo armband, and do the testing with the myo armband placed slightly elsewhere)
	
\end{itemize}
	
	
%	\begin{table}[h]
%		\caption{An Example of a Table}
%		\label{table_example}
%		\begin{center}
%			\begin{tabular}{|c||c|}
%				\hline
%				One & Two\\
%				\hline
%				Three & Four\\
%				\hline
%			\end{tabular}
%		\end{center}
%	\end{table}
	


	
	
	%Figure Labels: Use 8 point Times New Roman for Figure labels. Use words rather than symbols or abbreviations when writing Figure axis labels to avoid confusing the reader. As an example, write the quantity ÒMagnetizationÓ, or ÒMagnetization, MÓ, not just ÒMÓ. If including units in the label, present them within parentheses. Do not label axes only with units. In the example, write ÒMagnetization (A/m)Ó or ÒMagnetization {A[m(1)]}Ó, not just ÒA/mÓ. Do not label axes with a ratio of quantities and units. For example, write ÒTemperature (K)Ó, not ÒTemperature/K.Ó