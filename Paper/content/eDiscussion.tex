\textbf{Comparison of features.}
The online results indicated no significant difference between LogVar and MAV in the performance scores both with (p = 0.5637) and without (p = 0.0833) IMU data included. Based on a study \cite{hahne2014} showing LogVar as a feature with linear properties, it would be expected that this feature would perform better in a linear regression model, than a feature which to the authors knowledge has not been proven to be linear. On the contrary it was shown that a significantly higher number of targets was reached with a linear regression models based on the MAV feature with IMU included, compared to the LogVar regression model with IMU included (p = 0.0017). When IMU data was not included, there was no difference between the number of targets reached in the test (p = 1).

Further studies within this field should consider examining other features and studying the effect of combining several features in order to further improve performance independent of the limb position.

\textbf{Inclusion of IMU data.}
The IMU data included in this study was based on a single accelerometer, where it was expected that the Myo armband would give a similar output as long as the subjects were performing both training and testing from the same starting position. Inclusion of the IMU data was shown to yield the same results in the online test scores, with no significant difference for either MAV (p = 0.1779) or LogVar (p = 0.5637) when comparing regression models trained with and without accelerometer inputs. Inclusion of the IMU data yielded significantly poorer results for the LogVar regression model (p = 0.0016), while it led to a significant improvement of the MAV regression model (p = 0.0124) when examining the number of reached targets. The inclusion of IMU data could be a subject of further investigation, as the results might be improved by implementing a system capable of measuring the angles of the joints, in order to create a more versatile and usable regression model outside the clinical environment. Including IMU data could additionally be used to select specific regression models, if a system was build with models fitted for each limb position instead of the same regressors for all positions. 

\textbf{Stability in limb positions.}
When excluding IMU data, there was no significant difference between the performance score for either LogVar (p = 0.2359) or MAV (p = 0.8948) in the different limb positions, while there was a difference between the number of reached targets for MAV (0.0212) but no difference for LogVar (p = 0.4220). This outcome shows that both MAV and LogVar yields stable performance in different limb positions in a linear regression-based control scheme. This finding agrees with Hwang et al. \cite{Hwang2017}, who equivalently found stable online performance across limb positions in a linear regression-based control scheme applying RMS as feature.

When including IMU data the MAV based regression model was shown to have a significant difference in scores between limb positions (p = 0.0319), while LogVar did not (p = 0.4594). While the target reaching time were shown to be different depending on limb positions when using MAV, the number of targets reached was improved, as there was no significant difference (p = 0.2957) in the amount of targets reached. The LogVar feature based regression models were shown to have a difference between reached targets when using IMU data (p = 0.0037).

Overall the LogVar regression models were observed as being the most unstable in the different limb positions when examining the test subjects performance in the target-reaching test. This might be a result of the LogVar feature being based on the change of the signal, as this could lead to problems with crosstalk, when the arm is not in a relaxed state. MAV was observed as being more stable, with the subjects being able to create more smooth movements as well as being able to controllably return to the resting position. Based on the findings of this study, it would be recommended to examine features based on the amplitude rather than the variance in future studies within this area.


\textbf{Offline vs. online training.}
Offline testing was only done for MAV and LogVar without IMU data included. A significant difference between the two features when testing with training data (p = 0.0044) was archived in the offline test, but no significant difference when testing with new data (p = 0.1138). Comparing RMSE of LogVar with training data and RMSE of LogVar with new data there was a significant difference (P = 0.0001), where RMSE of the test with new data has the higher mean. Same results were yielded for the MAV trained regressors (P = 0.000005). This indicates that the regression models were overfitted when exposed to new data. The online results yielded robost control across all limb positions, and therefore no apparent correlation between offline and online testing. This could be caused by the subjects ability to adjust to a poor fitted model when given visual feedback while performing the target-reaching test. This observation corresponds to findings in another study \cite{jiang2010}.


\textbf{Limitations of the study.}
This study was based on data from 12 test subjects, where three had to be excluded. One subject was excluded due to misunderstanding the given instructions and thereby creating an unusable set of training and test data. This limited the control of the regression models giving the subject a mean score above 25 s per target reached and average number of reached targets below 10 for all tests.

Two other subjects were excluded as the recorded intensities were not high enough to differ between the baseline and the higher EMG intensity. This caused the regression models to interpret the baseline in the target-reaching test as movements being performed at between 30\% and 70\% of the MVC. 

To improve the validity of the findings more test subjects should be included in future studies within this field. Subjects with transradial amputations should also be taken into consideration if regression based control schemes were to be considered for future use in myoelectric prosthetic devices. 

Using the Myo armband for data acquisition limited the sampling rate to 200 Hz. Only the 0-100 Hz spectrum of the EMG was represented correctly, where frequencies above 100 Hz was affected by aliasing. Along with frequency representation limitations, the Myo armband restricted the number and placement of electrodes to eight channels placed at the same distance distal to the elbow joint, where it might be possible to yield better results with a different electrode placement and number of channels. Further studies should implement conventional EMG electrodes and an ADC with a sufficient sample rate, enabling the entire frequency band of EMG signals to be acquired correctly.


%\begin{itemize}
%	\item inclusion of more subjects
%	\item sample rate of the myo armband - exclusion of test subjects
%	\item even though LogVar shows linearity in a previous study, it does not perform better in linear regression than a presumably non-linear feature
%	\item Both features does not show significant difference in control in different limb positions
%	\item Whatever the inclusion of IMU shows
%	\item It is possible to use regression as control method to yield no significantly different performance in variations of limb positions
%	\item Use other regression methods and other features to analyse if it will result in better performance
%	\item why does the inclusion of IMU improve the performance
%	\item No connection between offline and online tests, which also is shown in previous studies
%	\item reasonable control can be archived when donning/doffing(we train the regressors with data, take of the myo armband, and do the testing with the myo armband placed slightly elsewhere)

%\end{itemize}


%discussion points
%\section{Discussion}
%The present paper had as porpuose to study the performance of linear regression methods for myoelectric prostheses control taking in acount the limb position effect. Along the process different aspects have been exposed that are important to mention.\\
%
%%	\subsection{Offline and online training}
%	%Offline and online training was done for both MAV and LogVar without IMU data included, with a significant difference between the two features when testing with training data (p = 0.0044) but no difference when testing with new data (p = 0.1138). Overall it was found that there were no apparent correlation between offline and online testing. This could be caused by the subjects ability to adjust to a poor fitted model when given the visual feedback while performing the target test. This finding corresponds to the findings in other studies \cite{jiang2010}.
%	
%	During the results examination it was found that there is no apparent correlation between the offline and online results for both features under study. 
%	%On the one hand the offline outcomes illustrates overfitting of the regression models. On the other hand the online test yielded robust control of the wrist movements performanced in the three different limb positions. 
%	This could be owing to the subjects$'$ ability to compensate for a poorer fitting model when visual feedback is provided while performing the target test. This conclusion corresponds to the findings in other study \cite{jiang2010}.
%	
%	%\subsection{Comparison of features}
%	%In the results it was found that there was no significant difference between the performance of LogVar and MAV when it comes to the scores for the target test both with (p = 0.5637) and without (p = 0.0833) IMU data included. Based on previous studies showing LogVar as a feature with linear properties, it would be expected that this feature would perform better in a linear regression model, than a feature which to the authors knowledge has not been proven to be linear. On the contrary it was shown that a significantly higher number of targets was reached with a linear regression models based on the MAV feature with IMU included, compared to the LogVar regression model with IMU included (p = 0.0017). When IMU data wasn't included, there was no difference between the number of targets reached in the test (p = 1).
%	
%	%Further studies within this field should consider examining other features, while studying the effect of combining several features in order to yield better performance independent of the limb position.
%	
%	%Based on prevoius studies two different features were extracted in order to tran the regressors MAV and LogVar.
%	There was found no significant difference between the performance of MAV and LogVar in the target test for both conditions, when IMU data is included and when is not. A study by Hanhe \cite{hanhe2014} et al. showed that LogVar presented linear properties and outperformed the other different features extracted for that particular research. It would be expected that this feature performed better in a linear regression model. However, the obtained results illustrates that MAV outperformed the LogVar in the number of reached targets when IMU was included. 
%	Further studies within this field should consider examining other features, while studying the effect of combining several features in order to yield better performance independent of the limb position.
%	
%	%\subsection{Inclusion of IMU data}
%	%The IMU data included in this study was based on a single accelerometer, where it was expected that the Myo band would give a similar output as long as the subjects were performing both training and testing from the same starting position. Inclusion of the IMU data was shown to yield the same results when it comes to the test score, with no significant difference for either MAV (p = 0.1779) or LogVar (p = 0.5637) when comparing regression models build with and without accelerometer inputs. It was found that the inclusion of the IMU data yielded significantly worse results for the LogVar regression model (p = 0.0016), while it led to a significant improvement of the MAV regression model (p = 0.0124) when examining the number of reached targets. The inclusion of IMU data could be a subject of further investigation, as the results might be improved by implementing a system capable of measuring the angles of the joints, in order to create a more versatile and usable regression model outside the clinical environment. %Conclusion?? 
%	
%	The inclusion of IMU data proved no significant difference for both features in the test score. However the inclusion of IMU data led in better results in the number of reached targets for MAV (p = 0.0124) than LogVar (p = 0.0016). It was expected that the inclusion of IMU for this study would give similar output since the input data was based on a single accelerometer and the performance was executed in the same position. The inclusion of IMU data could be a subject of further investigation, as the results might be improved by implementing a system capable of measuring the angles of the joints, in order to create a more versatile and usable regression model outside the clinical environment.
%	
%	%\subsection{Stability in limb positions}
%	%When excluding IMU data, there was no significant difference between the target score for either LogVar (p = 0.2359) or MAV (p = 0.8948) in the different limb positions, while there was a difference between the number of reached targets for MAV (0.0212) but no difference for LogVar (p = 0.4220). This outcome shows that both MAV and LogVar yields rather stable performance in different limb positions when used to create linear regression based control schemes.
%	
%	%When including IMU data the MAV based regression model was shown to have a significant difference between the score of different limb positions (p = 0.0319), while LogVar did not show any difference (p = 0.4594). While the time taken to reach targets were shown to be different depending on limb positions when using MAV, the number of targets reached was improved, so that the number of reached targets with MAV were not shown to be significantly different (p = 0.2957). The LogVar feature based regression models were shown to have a difference between reached targets when using IMU data (p = 0.0037).
%	
%	%Overall the LogVar regression models were observed as being the most unstable in the different limb positions when looking at the test subjects performance in the target test. This might be a result of the LogVar feature being based on the change of the signal, as this could lead to problems with crosstalk when the arm is not in a relaxed state. The MAV was observed as being more stable, with the subjects being able to create more controlled movements as well as having the possibility to adjust the position of the arrow when trying to get back to the middle of the test GUI. Based on the findings of this study, it would be recommended to examine features based on the amplitude rather than the variance in future studies within this area.
%	
%	There was no significant difference in target score for both features in different limb positions when IMU was not included.  However, the results present a difference between the amount of reached targets for MAV (p=0.0212) but not for LogVar (p = 0.4220). The inclusion of IMU data showed a significant difference between the score depending on the limb position for MAV (p = 0.0319), but no difference was shown for LogVar (p = 0.4594). The LogVar feature based regression models were shown to have a difference between reached targets when using IMU data (p = 0.0037).
%	 
%	Overall the LogVar regression models were observed as being the most unstable in the different limb positions when looking at the test subjects performance in the target test. This might be a result of the LogVar feature being based on the change of the signal, as this could lead to problems with crosstalk when the arm is not in a relaxed state. The MAV was observed as being more stable, with the subjects being able to create more controlled movements as well as having the possibility to adjust the position of the arrow when trying to get back to the middle of the test GUI. Based on the findings of this study, it would be recommended to examine features based on the amplitude rather than the variance in future studies within this area.
%	%Futhermore online results were not siginificant different depending on the limb position as other studies had shown for classification control scheme. This findings agree with a recently published study by Hwang et al. \cite{Hwang2017}. In their study different limb position were tested using Root Mean Square (RMS) as feature to train the regressors.
%	
%	%\subsection{Limitations of the study}
%	%This study was based on data from 12 test subjects, where three had to be excluded. One subject was excluded due to misunderstanding the given instructions and thereby creating an unusable set of training and test data, which limited the control of the regression models giving him a mean score above 25 seconds per target and average number of reached targets below 10 for all tests.
%	
%	%Two other subjects were excluded as the recorded intensities were not high enough to differ between the baseline and the intended movement. This caused the regression models to interpret the baseline in the target test as movements being performed at between 30\% and 70\% of the MVC.
%	
%	%To improve the validity of the study more test subjects should be included in further studies within this field. Subjects with transradial amputations should also be taken into consideration if the regression based control schemes were to be considered for future use in myoelectric prosthetic devices. 
%	
%	%Using the Myo band for data acquisition led to certain limitations in the sample rate, as the device is only capable of recording signals between 0 and 200Hz. This leads to the final signal being recorded between 0 and 100Hz, where everything above 100Hz is affected by aliasing. Along with frequency limitations, the Myo band restricted the number and placement of electrodes to eight channels placed at the same distance from the elbow, where it might be possible to yield better results with a different electrode placement and number. Further studies should implement regular EMG electrodes in order to obtain a more usable and clear signal, that represents the entire frequency band of EMG signals.
%	
%	During the accomplishment of this study some of the subjects were not able to perform consistently throughout the data set. In order to ensure constant data those subjects were exclude. Even though the final results present that is possible to achieve simultaneous and proportional control using Myo armband, it had been shown that the sampling rate of this device was a limitation on the acquisition of sEMG data. This could be due to the fact that sEMG signals rate is between 10-500Hz, however the sampling rate of the Myoarmband is 200Hz, this fact leads on the sEMG signals above 100Hz would be affected by aliasing. Furthermore the quantity and placement of electrodes was restricted because of the Myo armband. It is believed that the implementation of regular EMG electrodes could elicit better results. To improve the validity of the study more test subjects should be included in further studies within this field. Subjects with transradial amputations should also be taken into consideration if the regression based control schemes were to be considered for future use in myoelectric prosthetic devices. 
%	
%%	\begin{table}[h]
%%		\caption{An Example of a Table}
%%		\label{table_example}
%%		\begin{center}
%%			\begin{tabular}{|c||c|}
%%				\hline
%%				One & Two\\
%%				\hline
%%				Three & Four\\
%%				\hline
%%			\end{tabular}
%%		\end{center}
%%	\end{table}
%	
%
%
%	
%	
%	%Figure Labels: Use 8 point Times New Roman for Figure labels. Use words rather than symbols or abbreviations when writing Figure axis labels to avoid confusing the reader. As an example, write the quantity ÒMagnetizationÓ, or ÒMagnetization, MÓ, not just ÒMÓ. If including units in the label, present them within parentheses. Do not label axes only with units. In the example, write ÒMagnetization (A/m)Ó or ÒMagnetization {A[m(1)]}Ó, not just ÒA/mÓ. Do not label axes with a ratio of quantities and units. For example, write ÒTemperature (K)Ó, not ÒTemperature/K.Ó