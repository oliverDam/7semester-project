something!!! 
	
	\subsection{Separability of the data} 
	%PCA performance for both features extracted is illustrated in Fig. \ref{PCA_MVA} (MVA) and in Fig. \ref{PCA_logvar} (logarithmic variance).
	%The importance of each component for MAV and logarithmic variance is represented in Fig. \ref{PCA_MVA}(a) and Fig. \ref{PCA_logvar}(a) respectively. On the one hand for the MAV feature making use of the three first components, 93.8\% of the data set can be described. On the other hand for the logarithmic variance feature the 93.97\% of the data set is describe with the three first components. This three PC have been plot for both features Fig. \ref{PCA_MVA}(b) and Fig. \ref{PCA_logvar}(b). There is no presence of remarkable outliers on two data sets and the formed clusters are easily distinguishable.


	\subsection{Regression accuracy} 
%Through a quantitative examination of the figure(figure),is observed that MAV performed slightly better than the logarithmic variance for low intensities. However, both estimates yielded inaccurate fitting in high intensities, especially for ulnar deviation of the wrist. It has been extracted from the results illustrated in (figure), which represent the RMSE of the four different regressos for both features of the training data, than $RMSE_{MAV}$= 0.0867$\pm 0.031$ and $RMSE_{logvar}$= 0.1047$\pm 0.0273$. Overall MAV provided a lower mean but a higher standard deviation compared with logarithmic variance.

		%\begin{figure}[thpb]
		%\centering
		%\includegraphics[scale=.27]{Figures/pcasubplotMAV}
		%\caption{Inductance of oscillation winding on amorphous
		%	magnetic core versus DC bias magnetic field}
		%\label{PCA_MVA}
	%\end{figure}

