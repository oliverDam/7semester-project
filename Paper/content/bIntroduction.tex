In recent years the development of EMG controlled prosthesis have advanced due to an increased interest in the area as well as a higher demand of better control of this prosthesis.\cite{fougner2012}
In the early years most EMG prosthetics functioned by only controlling one DOF by \textit{on-off control}, mostly by linking antagonistic muscles to one DOF. %This along with \textit{mode switching} provided users a way to control more than one DOF, but not in a simultaneous way. 
This control provided a way to control more than one DOF, but not simultaneously. However, as demands would rise, more complex methods were introduced to the EMG scene. Classification methods enabled simultaneous control of more than one DOF, but gave rise to new problems; a wider range of control would give less accurate movements, and training the pattern recognition methods proved difficult, as the training could over-fit, causing extended use of the prosthetics to degrade in performance. \cite{Ison2016}.
 
It has been proved that regression techniques can be apply as a new mapping method to achieve simultaneous and proportional control of multiple DOFs\cite{hanhe2014}. Regression methods provide a continous value for each DOF based on the recorded EMG signal, while classification methods only decides upon a certain class. However there are still difficulties when prosthesis perform outside the clinical training environment\cite{jiang2012}.
 Fougner et al.\cite{Fougner2011} noticed that majority of studies only take in account one limb position. It has been shown that some muscles are activated based on joint angles \cite{reference}, even though the muscles are not involved in the movement of that joint, which can be explained by muscle-synergies existed between muscles.
% which becomes a problem since muscles create muscle-synergies to perform movements. 
Variations in limb positions can have an impact on the robustness of EMG pattern recognition. %To be able to offer a good perform of the prosthesis, these should be able to execute with the same accuracy diverse hand gestures in different limb positions. 
In order to overcome this problem it has been suggested to combine EMG data as well as IMU data, to provide limb position information. Nevertheless this combination of data have only been investigated for classification methods. 
In this study we test the performance of linear regression methods combining EMG and IMU data for a simultaneous and proportional control of two DOFs in a lower-arm prothesis while the arm is located in different limb positions.
% in the training sesions of the regressor to obtain simultaneous and proportional control of EMG prosthesis. 
%Hypotheses
%Simultaneous and proportional control of two  DOF's of the wrist in different limb positions, can be achieve trough the use of linear regression as control system. Combining EMG and IMU's can minimize the limb position effect when using regression as control system.