%What has been done in the EMG-prosthesis area?
In recent years the development of EMG controlled lower arm prosthetics has advanced considerably, due to an increased interest in the area along with higher demands for better prosthetics and more precise control. \cite{Fougner2012} In the early years most EMG prosthetics functioned by controlling one degree of freedom (DOF) with on-off control, mainly by linking antagonistic muscles to one DOF. This kind of prostheses change between states due to a switching impulse which cause a state machine to shift its present state. Usually a strong and fast muscle contraction from the users is employed to generate the switching signals. \cite{amsuess2014}
This type of control provided users a way to control more than one DOF, but never simultaneously. The switch-control functioned on a cycle, so users would have to go through all the movements of the prosthesis to find the one they wanted to perform. However, as demands rose, more complex methods was introduced to the EMG prosthetics scene. Classification methods effectively enabled users to use DOF's more freely because the switching was now replaced by direct recognition of different muscle contractions linked to specific prosthetic movements. This also effectively enabled proportional control of movements, but gave rise to new problems: a wider range of control would give less accurate movements, and training the classifiers proved difficult, as the training could over-fit, causing extended use of the prosthetics to degrade in performance. \cite{Ison2016}

%Regression
Introducing regression as a new mapping method in myoelectric prosthetics provided a way to enable both simultaneous and proportional control of multiple DOF's. Regression is able to provide a continuous value for each DOF based on the recorded EMG signal, while a classifier only decides upon a certain class. \cite{hahne2014, jiang2010}
This means that classification can only translate a recorded EMG signal to one movement of the prosthetic at a time. It can do so proportionally but the handling still lacks natural control, since movements by able-bodied individuals very rarely only happen in one DOF at a time. Regression methods constantly provide a value, and since several regressors can be used at a time, several values can be used in the recognition of movements. This is what enables regression methods to perform simultaneous and proportionally. 

Applying regression as a mapping method in proportional and simultaneous control of multiple DOF's has been shown to perform well in recognition of movements and doing so with a low computation time. \cite{hahne2014} However, very few studies have tested the regressor performance in daily life tasks outside the clinical training environment. \cite{jiang2012} A study by Fougner et al. \cite{Fougner2011} has addressed the problem that most studies test their method on only one limb position. This means that the actual performance of regression methods has not yet been properly addressed when recognizing movements, where the arm changes position during daily life tasks. 

%Which issues are there in the EMG-prosthesis area?(decreasing quality of control of hand gestures when the arm is placed in different positions)
When recording EMG signals it has been shown that some muscles are activated based on joint angles, even though the muscles are not involved in the movement of that joint \cite{Fougner2011}. This provides a problem, but can be explained by muscle-synergies \cite{DeRugy2013}. These muscle-synergies are created by the Central Nervous System (CNS) and coordinated into activation of different muscles at varying times. This enables the CNS to control the muscle-synergies instead of controlling each muscle individually to perform movements \cite{jiang2009}. This means that muscles in the lower arm can be activated when muscles in the upper arm are activated, in a level that will be detectable in EMG recordings, and enough to alter recognition of movements, when the arm is active in limb positions other than those tested in a clinical environment. 

%What would be novel to add to this area?(adding IMU’s to a regressor, since it has been done with a classifier)
In order to overcome the problem of muscles-synergies, Fougner et al. \cite{Fougner2011} has suggested to combine recordings of EMG signals with inertial information to provide limb position data. This could be beneficial in increasing the accuracy of EMG controlled prosthetics for use in daily life tasks. 
Even though the combination of EMG and Inertial Measurement Unit (IMU) data has been proposed as a valid way to improve the performance and accuracy of EMG based prosthetics, it has only been investigated in few studies. \cite{Roy2010, Imtiaz2014, jiang2012}

To the authors knowledge the combination of EMG recordings and IMU data has only been done with classification methods. A novel approach to further investigate the usability of combining EMG and IMU is to build a regression based control scheme for myoelectric prosthetics. This would enable both proportional and simultaneous control of several DOF's, where the inclusion of IMU data should provide more information on limb position to counter the effect of limb position.






%In recent years the development of EMG controlled prosthesis have advanced due to an increased interest in the area as well as a higher demand of better control of this prosthesis.\cite{fougner2012}
%In the early years most EMG prosthetics functioned by only controlling one DOF by \textit{on-off control}, mostly by linking antagonistic muscles to one DOF. %This along with \textit{mode switching} provided users a way to control more than one DOF, but not in a simultaneous way. 
%This control provided a way to control more than one DOF, but not simultaneously. However, as demands would rise, more complex methods were introduced to the EMG scene. Classification methods enabled simultaneous control of more than one DOF, but gave rise to new problems; a wider range of control would give less accurate movements, and training the pattern recognition methods proved difficult, as the training could over-fit, causing extended use of the prosthetics to degrade in performance. \cite{Ison2016}.
% 
%It has been proved that regression techniques can be apply as a new mapping method to achieve simultaneous and proportional control of multiple DOFs\cite{hanhe2014}. Regression methods provide a continous value for each DOF based on the recorded EMG signal, while classification methods only decides upon a certain class. However there are still difficulties when prosthesis perform outside the clinical training environment\cite{jiang2012}.
% Fougner et al.\cite{Fougner2011} noticed that majority of studies only take in account one limb position. It has been shown that some muscles are activated based on joint angles \cite{reference}, even though the muscles are not involved in the movement of that joint, which can be explained by muscle-synergies existed between muscles.
%% which becomes a problem since muscles create muscle-synergies to perform movements. 
%Variations in limb positions can have an impact on the robustness of EMG pattern recognition. %To be able to offer a good perform of the prosthesis, these should be able to execute with the same accuracy diverse hand gestures in different limb positions. 
%In order to overcome this problem it has been suggested to combine EMG data as well as IMU data, to provide limb position information. Nevertheless this combination of data have only been investigated for classification methods. 
%In this study we test the performance of linear regression methods combining EMG and IMU data for a simultaneous and proportional control of two DOFs in a lower-arm prothesis while the arm is located in different limb positions.
%% in the training sesions of the regressor to obtain simultaneous and proportional control of EMG prosthesis. 
%%Hypotheses
%%Simultaneous and proportional control of two  DOF's of the wrist in different limb positions, can be achieve trough the use of linear regression as control system. Combining EMG and IMU's can minimize the limb position effect when using regression as control system.