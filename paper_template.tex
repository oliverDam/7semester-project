%%%%%%%%%%%%%%%%%%%%%%%%%%%%%%%%%%%%%%%%%%%%%%%%%%%%%%%%%%%%%%%%%%%%%%%%%%%%%%%%
%2345678901234567890123456789012345678901234567890123456789012345678901234567890
%        1         2         3         4         5         6         7         8

%\documentclass[letterpaper, 10 pt, conference]{ieeeconf}  % Comment this line out
% if you need a4paper
\documentclass[a4paper, 10pt, conference]{ieeeconf}      % Use this line for a4
% paper



% The following packages can be found on http:\\www.ctan.org
%\usepackage{graphics} % for pdf, bitmapped graphics files
%\usepackage{epsfig} % for postscript graphics files
%\usepackage{mathptmx} % assumes new font selection scheme installed
%\usepackage{times} % assumes new font selection scheme installed
%\usepackage{amsmath} % assumes amsmath package installed
%\usepackage{amssymb}  % assumes amsmath package installed

\title{\LARGE \bf
	Simultaneous and proportional myoelectric control
}%final name

%\author{ \parbox{3 in}{\centering Huibert Kwakernaak*
%         \thanks{*Use the $\backslash$thanks command to put information here}\\
%         Faculty of Electrical Engineering, Mathematics and Computer Science\\
%         University of Twente\\
%         7500 AE Enschede, The Netherlands\\
%         {\tt\small h.kwakernaak@autsubmit.com}}
%         \hspace*{ 0.5 in}
%         \parbox{3 in}{ \centering Pradeep Misra**
%         \thanks{**The footnote marks may be inserted manually}\\
%        Department of Electrical Engineering \\
%         Wright State University\\
%         Dayton, OH 45435, USA\\
%         {\tt\small pmisra@cs.wright.edu}}
%}

\author{Simon Bruun, Oliver Thomsen Damsgaard, Martin Alexander Garenfeld, Irene Uriarte}% <-this % stops a space


\begin{document}
	
	
	
	\maketitle
	\thispagestyle{empty}
	\pagestyle{empty}
	
	
	%%%%%%%%%%%%%%%%%%%%%%%%%%%%%%%%%%%%%%%%%%%%%%%%%%%%%%%%%%%%%%%%%%%%%%%%%%%%%%%%
	\begin{abstract}
Electromyography (EMG) is widely used as input to control scheme of myoelectric prosthetics. However, EMG signals change with limb position and thus lowers the accuracy in classification.Inclusion of the use of Inertial Measurement Units (IMU) has proved to raise the accuracy in pattern recognition methods. However, pattern recognition methods provides only control of one degree of freedom (DoF) at a time, and are computational costly. This study propose to use the combination of EMG recordings and accelerometer data in a linear regression model to overcome the slower reaction time of pattern recognition systems and to enable a simultaneous and proportional control scheme. In this study recordings from four able-bodied subjects has been collected, performing four wrist movements in three different limb positions. The data is evaluated through principal component analysis (PCA) and processed/trained with a linear regression model to classify the hand movements. One regressor is trained for each hand movement, using EMG data as well as a combination of EMG data and IMU data. The regressors are tested in a real-time visual environment on PC, measuring time to complete a target-reaching task of eight targets. The performance of the regressors are compared between using the IMU data and not using IMU data to determine the effect of including IMU data. 
	
	\end{abstract}
	
	
	%%%%%%%%%%%%%%%%%%%%%%%%%%%%%%%%%%%%%%%%%%%%%%%%%%%%%%%%%%%%%%%%%%%%%%%%%%%%%%%%
	\section{INTRODUCTION}
In recent years the development of EMG controlled prosthesis have advanced due to an increased interest in the area as well as a higher demand of better control of this prosthesis.\cite{fougner2012}
In the early years most EMG prosthetics functioned by only controlling one DOF by \textit{on-off control}, mostly by linking antagonistic muscles to one DOF. This along with \textit{mode switching} provided users a way to control more than one DOF, but not in a simultaneous way. However, as demands would rise, more complex methods was introduced to the EMG scene, and proportional control was brought in with pattern recognition methods. This effectively enabled simultaneous control of more than one DOF, but gave rise to new problems; a wider range of control would give less accurate movements, and training the pattern recognition methods proved difficult, as the training could over-fit, causing extended use of the prosthetics to degrade in performance. \cite{Ison2016}. It has been proved that regression techniques can be apply as a new mapping method to achieve simultaneous and proportional control of multiple DOFs\cite{hanhe2014}. However there are still difficulties when prosthesis perform outside the clinical training environment\cite{jiang2012}. Fougner et al.\cite{Fougner2011} noticed that majority of studies only take in account one limb position which becomes a problem since muscles create muscle-synergies to perform movements. It has been demonstrated that the variations in limb positions can have an impact on the robustness of EMG pattern recognition. In order to overcome this problem it has been suggested to combine EMG data as well as IMU data in the training sesions of the regressor to obtain simultaneous and proportional control of EMG prosthesis.
%Hypothesis
Simultaneous and proportional control of two DOF's of the wrist in different limb positions, can be achieve trough the use of linear regression as control system. Combining EMG and IMU's can minimize the limb position effect when using regression as control system.	

	\section{METHODS}
	
	\subsection{Experimental Setup}
EMG data was collected from four able-bodied subjects (three males, one female). The subjects performed four different hand gestures. This study is only focus on two DOF, which are, flexion and extension, radial and ulnar deviation of the wrist. The order in the execution of the movements was the same for each subject. EMG signals were recorded with Myo armband, this device counts with eight medical grade stainless steel surface EMG sensors. Furthermore its nine axis IMU provides information about position and orientation of the arm. The Myo armband was positioned in the right forearm of the subjects (all subjects right handed). The procedure was performed in three different limb positions. In order to avoid shoukder fatigue a relaxation period was given between trials. The subjects were instructed not to move the fingers during the data acquisition. The process was performed in a standing position. \\
 
%In order to acquire the training data a graphical user interface (GUI) was implemented in MATLAB. Firstly baseline was meassure holding the forearm relaxed and the wrist in a neutral position for the correspond limb position. This measure was substracted from the signal afterwards to be able to remove the present artefacts. 

%baseline,MVC...
	
	%\subsection{Maintaining the Integrity of the Specifications}
	
	%The template is used to format your paper and style the text. All margins, column widths, line spaces, and text fonts are prescribed; please do not alter them. You may note peculiarities. For example, the head margin in this template measures proportionately more than is customary. This measurement and others are deliberate, using specifications that anticipate your paper as one part of the entire proceedings, and not as an independent document. Please do not revise any of the current designations
	
	\subsection{Preprocessing}
	For this study the EMG data were filtered using a second-order Butterworth high-pass filter, cutoff frequency ($f_c$=10Hz). 
	\subsection{Feature extraction}
	The features were extracted creating a sliding-window of 40 samples with an overlapping of the 50\%. Two different time domain features were extracted, Mean absolute value (MAV) as well as logarithmic variance. MAV represent the amplitud of the signal. It is defined as the average of the absolute values of the EMG signal and expresed as:
	\begin{equation}
	MAV = \frac{1}{N}\sum\limits_{i=1}^N|x_i|
	\end{equation}
	where N is the length of the signal, and $x_i$ is the signal of $i$ samples.
	The logarithmic variance is a nonlinear transformation of the variance applied to %linealidad
	\begin{equation} \label{eq:logvar}
	log(\sigma^2) = log(\frac{\sum\limits_{i=1}^N(x_i - \mu)^2}{N})
	\end{equation}
	where N expresses the length of the signal, $x_i$ is the $i^th$ sample of the signal and $\mu$ is the mean.
	\subsection{Regression models}
	The acquired data was used to train the four different regressors that had been implemented, one for each of the movements under study. Simple linear regression had been applied as is shown in \ref{reg}:
\begin{equation} \label{eq:simpleLinearRegression}
Y_i = \alpha + \beta X_i + \epsilon_i
\label{reg}
\end{equation}
where, $Y$ is the dependent variable or response, $X$ is the independent variable or the predictor, $\beta$ is the regression coefficient or the slope, and $\alpha$ is the Y intercept (predicted value of $Y$ at $X = 0$),  $\epsilon$ is the error and $i$ is the index.
	\subsection{Separability of data}
	In order to evaluate the quality of the features extracted from the EMG signals, Principal Component Analysis (PCA) was applied. This analysis tool was performed for each movement in each limb position and represented in three dimensional space as shown in %figure. 
	Through this tool is  possible to see significant outliers or if the clusters formed by the features can be easily distinguishable. This was done to avoid inaccurate training of the regressors.
\begin{figure}[thpb]
	\centering
	\framebox{\parbox{3in}{We suggest that you use a text box to insert a graphic (which is ideally a 300 dpi TIFF or EPS file, with all fonts embedded) because, in an document, this method is somewhat more stable than directly inserting a picture.
	}}
	%\includegraphics[scale=1.0]{figurefile}
	\caption{Inductance of oscillation winding on amorphous
		magnetic core versus DC bias magnetic field}
	\label{figurelabel}
\end{figure}
	\subsection{Regressor accuracy}
	To meassure the accuracy of the regressor, Root Mean Squared Error (RMSE) was calculated. RMSE is a calculation of the standard deviation of the residuals, that is, the difference between the estimated and the actual values.
	%equation
	\begin{equation}
	RMSE = \sqrt{\frac{\sum\limits_{i=1}^N(y_i - \hat{y_i})^2}{N}}
	\end{equation}
	Where N is the length of the signal, $y_i$ is the $i^th$ variable of the actual data and $\hat{y_i}$ is the $i^th$ output of the regressor. The RMSE will be done for the regressor of each movement.
	\section{MATH}
	
	Before you begin to format your paper, first write and save the content as a separate text file. Keep your text and graphic files separate until after the text has been formatted and styled. Do not use hard tabs, and limit use of hard returns to only one return at the end of a paragraph. Do not add any kind of pagination anywhere in the paper. Do not number text heads-the template will do that for you.
	
	Finally, complete content and organizational editing before formatting. Please take note of the following items when proofreading spelling and grammar:
	
	\subsection{Abbreviations and Acronyms} Define abbreviations and acronyms the first time they are used in the text, even after they have been defined in the abstract. Abbreviations such as IEEE, SI, MKS, CGS, sc, dc, and rms do not have to be defined. Do not use abbreviations in the title or heads unless they are unavoidable.
	
	\subsection{Units}
	
	\begin{itemize}
		
		\item Use either SI (MKS) or CGS as primary units. (SI units are encouraged.) English units may be used as secondary units (in parentheses). An exception would be the use of English units as identifiers in trade, such as Ò3.5-inch disk driveÓ.
		\item Avoid combining SI and CGS units, such as current in amperes and magnetic field in oersteds. This often leads to confusion because equations do not balance dimensionally. If you must use mixed units, clearly state the units for each quantity that you use in an equation.
		\item Do not mix complete spellings and abbreviations of units: ÒWb/m2Ó or Òwebers per square meterÓ, not Òwebers/m2Ó.  Spell out units when they appear in text: Ò. . . a few henriesÓ, not Ò. . . a few HÓ.
		\item Use a zero before decimal points: Ò0.25Ó, not Ò.25Ó. Use Òcm3Ó, not ÒccÓ. (bullet list)
		
	\end{itemize}
	
	
	\subsection{Equations}
	
	The equations are an exception to the prescribed specifications of this template. You will need to determine whether or not your equation should be typed using either the Times New Roman or the Symbol font (please no other font). To create multileveled equations, it may be necessary to treat the equation as a graphic and insert it into the text after your paper is styled. Number equations consecutively. Equation numbers, within parentheses, are to position flush right, as in (1), using a right tab stop. To make your equations more compact, you may use the solidus ( / ), the exp function, or appropriate exponents. Italicize Roman symbols for quantities and variables, but not Greek symbols. Use a long dash rather than a hyphen for a minus sign. Punctuate equations with commas or periods when they are part of a sentence, as in
	
	$$
	\alpha + \beta = \chi \eqno{(1)}
	$$
	
	Note that the equation is centered using a center tab stop. Be sure that the symbols in your equation have been defined before or immediately following the equation. Use Ò(1)Ó, not ÒEq. (1)Ó or Òequation (1)Ó, except at the beginning of a sentence: ÒEquation (1) is . . .Ó
	
	\subsection{Some Common Mistakes}
	\begin{itemize}
		
		
		\item The word ÒdataÓ is plural, not singular.
		\item The subscript for the permeability of vacuum ?0, and other common scientific constants, is zero with subscript formatting, not a lowercase letter ÒoÓ.
		\item In American English, commas, semi-/colons, periods, question and exclamation marks are located within quotation marks only when a complete thought or name is cited, such as a title or full quotation. When quotation marks are used, instead of a bold or italic typeface, to highlight a word or phrase, punctuation should appear outside of the quotation marks. A parenthetical phrase or statement at the end of a sentence is punctuated outside of the closing parenthesis (like this). (A parenthetical sentence is punctuated within the parentheses.)
		\item A graph within a graph is an ÒinsetÓ, not an ÒinsertÓ. The word alternatively is preferred to the word ÒalternatelyÓ (unless you really mean something that alternates).
		\item Do not use the word ÒessentiallyÓ to mean ÒapproximatelyÓ or ÒeffectivelyÓ.
		\item In your paper title, if the words Òthat usesÓ can accurately replace the word ÒusingÓ, capitalize the ÒuÓ; if not, keep using lower-cased.
		\item Be aware of the different meanings of the homophones ÒaffectÓ and ÒeffectÓ, ÒcomplementÓ and ÒcomplimentÓ, ÒdiscreetÓ and ÒdiscreteÓ, ÒprincipalÓ and ÒprincipleÓ.
		\item Do not confuse ÒimplyÓ and ÒinferÓ.
		\item The prefix ÒnonÓ is not a word; it should be joined to the word it modifies, usually without a hyphen.
		\item There is no period after the ÒetÓ in the Latin abbreviation Òet al.Ó.
		\item The abbreviation Òi.e.Ó means Òthat isÓ, and the abbreviation Òe.g.Ó means Òfor exampleÓ.
		
	\end{itemize}
	
	
	\section{USING THE TEMPLATE}
	
	Use this sample document as your LaTeX source file to create your document. Save this file as {\bf root.tex}. You have to make sure to use the cls file that came with this distribution. If you use a different style file, you cannot expect to get required margins. Note also that when you are creating your out PDF file, the source file is only part of the equation. {\it Your \TeX\ $\rightarrow$ PDF filter determines the output file size. Even if you make all the specifications to output a letter file in the source - if you filter is set to produce A4, you will only get A4 output. }
	
	It is impossible to account for all possible situation, one would encounter using \TeX. If you are using multiple \TeX\ files you must make sure that the ``MAIN`` source file is called root.tex - this is particularly important if your conference is using PaperPlaza's built in \TeX\ to PDF conversion tool.
	
	\subsection{Headings, etc}
	
	Text heads organize the topics on a relational, hierarchical basis. For example, the paper title is the primary text head because all subsequent material relates and elaborates on this one topic. If there are two or more sub-topics, the next level head (uppercase Roman numerals) should be used and, conversely, if there are not at least two sub-topics, then no subheads should be introduced. Styles named ÒHeading 1Ó, ÒHeading 2Ó, ÒHeading 3Ó, and ÒHeading 4Ó are prescribed.
	
	\subsection{Figures and Tables}
	
	Positioning Figures and Tables: Place figures and tables at the top and bottom of columns. Avoid placing them in the middle of columns. Large figures and tables may span across both columns. Figure captions should be below the figures; table heads should appear above the tables. Insert figures and tables after they are cited in the text. Use the abbreviation ÒFig. 1Ó, even at the beginning of a sentence.
	
	\begin{table}[h]
		\caption{An Example of a Table}
		\label{table_example}
		\begin{center}
			\begin{tabular}{|c||c|}
				\hline
				One & Two\\
				\hline
				Three & Four\\
				\hline
			\end{tabular}
		\end{center}
	\end{table}
	
	
	\begin{figure}[thpb]
		\centering
		\framebox{\parbox{3in}{We suggest that you use a text box to insert a graphic (which is ideally a 300 dpi TIFF or EPS file, with all fonts embedded) because, in an document, this method is somewhat more stable than directly inserting a picture.
		}}
		%\includegraphics[scale=1.0]{figurefile}
		\caption{Inductance of oscillation winding on amorphous
			magnetic core versus DC bias magnetic field}
		\label{figurelabel}
	\end{figure}
	
	
	Figure Labels: Use 8 point Times New Roman for Figure labels. Use words rather than symbols or abbreviations when writing Figure axis labels to avoid confusing the reader. As an example, write the quantity ÒMagnetizationÓ, or ÒMagnetization, MÓ, not just ÒMÓ. If including units in the label, present them within parentheses. Do not label axes only with units. In the example, write ÒMagnetization (A/m)Ó or ÒMagnetization {A[m(1)]}Ó, not just ÒA/mÓ. Do not label axes with a ratio of quantities and units. For example, write ÒTemperature (K)Ó, not ÒTemperature/K.Ó
	
	\section{CONCLUSIONS}
	
	A conclusion section is not required. Although a conclusion may review the main points of the paper, do not replicate the abstract as the conclusion. A conclusion might elaborate on the importance of the work or suggest applications and extensions. 
	
	\addtolength{\textheight}{-12cm}   % This command serves to balance the column lengths
	% on the last page of the document manually. It shortens
	% the textheight of the last page by a suitable amount.
	% This command does not take effect until the next page
	% so it should come on the page before the last. Make
	% sure that you do not shorten the textheight too much.
	
	%%%%%%%%%%%%%%%%%%%%%%%%%%%%%%%%%%%%%%%%%%%%%%%%%%%%%%%%%%%%%%%%%%%%%%%%%%%%%%%%
	
	
	
	%%%%%%%%%%%%%%%%%%%%%%%%%%%%%%%%%%%%%%%%%%%%%%%%%%%%%%%%%%%%%%%%%%%%%%%%%%%%%%%%
	
	
	
	%%%%%%%%%%%%%%%%%%%%%%%%%%%%%%%%%%%%%%%%%%%%%%%%%%%%%%%%%%%%%%%%%%%%%%%%%%%%%%%%
	\section*{APPENDIX}
	
	Appendixes should appear before the acknowledgment.
	
	\section*{ACKNOWLEDGMENT}
	
	The preferred spelling of the word ÒacknowledgmentÓ in America is without an ÒeÓ after the ÒgÓ. Avoid the stilted expression, ÒOne of us (R. B. G.) thanks . . .Ó  Instead, try ÒR. B. G. thanksÓ. Put sponsor acknowledgments in the unnumbered footnote on the first page.
	
	
	
	%%%%%%%%%%%%%%%%%%%%%%%%%%%%%%%%%%%%%%%%%%%%%%%%%%%%%%%%%%%%%%%%%%%%%%%%%%%%%%%%
	
	References are important to the reader; therefore, each citation must be complete and correct. If at all possible, references should be commonly available publications.
	
	
	
	\begin{thebibliography}{99}
		
		\bibitem{c1} G. O. Young, ÒSynthetic structure of industrial plastics (Book style with paper title and editor),Ó 	in Plastics, 2nd ed. vol. 3, J. Peters, Ed.  New York: McGraw-Hill, 1964, pp. 15Ð64.
		\bibitem{c2} W.-K. Chen, Linear Networks and Systems (Book style).	Belmont, CA: Wadsworth, 1993, pp. 123Ð135.
		\bibitem{c3} H. Poor, An Introduction to Signal Detection and Estimation.   New York: Springer-Verlag, 1985, ch. 4.
		\bibitem{c4} B. Smith, ÒAn approach to graphs of linear forms (Unpublished work style),Ó unpublished.
		\bibitem{c5} E. H. Miller, ÒA note on reflector arrays (Periodical styleÑAccepted for publication),Ó IEEE Trans. Antennas Propagat., to be publised.
		\bibitem{c6} J. Wang, ÒFundamentals of erbium-doped fiber amplifiers arrays (Periodical styleÑSubmitted for publication),Ó IEEE J. Quantum Electron., submitted for publication.
		\bibitem{c7} C. J. Kaufman, Rocky Mountain Research Lab., Boulder, CO, private communication, May 1995.
		\bibitem{c8} Y. Yorozu, M. Hirano, K. Oka, and Y. Tagawa, ÒElectron spectroscopy studies on magneto-optical media and plastic substrate interfaces(Translation Journals style),Ó IEEE Transl. J. Magn.Jpn., vol. 2, Aug. 1987, pp. 740Ð741 [Dig. 9th Annu. Conf. Magnetics Japan, 1982, p. 301].
		\bibitem{c9} M. Young, The Techincal Writers Handbook.  Mill Valley, CA: University Science, 1989.
		\bibitem{c10} J. U. Duncombe, ÒInfrared navigationÑPart I: An assessment of feasibility (Periodical style),Ó IEEE Trans. Electron Devices, vol. ED-11, pp. 34Ð39, Jan. 1959.
		\bibitem{c11} S. Chen, B. Mulgrew, and P. M. Grant, ÒA clustering technique for digital communications channel equalization using radial basis function networks,Ó IEEE Trans. Neural Networks, vol. 4, pp. 570Ð578, July 1993.
		\bibitem{c12} R. W. Lucky, ÒAutomatic equalization for digital communication,Ó Bell Syst. Tech. J., vol. 44, no. 4, pp. 547Ð588, Apr. 1965.
		\bibitem{c13} S. P. Bingulac, ÒOn the compatibility of adaptive controllers (Published Conference Proceedings style),Ó in Proc. 4th Annu. Allerton Conf. Circuits and Systems Theory, New York, 1994, pp. 8Ð16.
		\bibitem{c14} G. R. Faulhaber, ÒDesign of service systems with priority reservation,Ó in Conf. Rec. 1995 IEEE Int. Conf. Communications, pp. 3Ð8.
		\bibitem{c15} W. D. Doyle, ÒMagnetization reversal in films with biaxial anisotropy,Ó in 1987 Proc. INTERMAG Conf., pp. 2.2-1Ð2.2-6.
		\bibitem{c16} G. W. Juette and L. E. Zeffanella, ÒRadio noise currents n short sections on bundle conductors (Presented Conference Paper style),Ó presented at the IEEE Summer power Meeting, Dallas, TX, June 22Ð27, 1990, Paper 90 SM 690-0 PWRS.
		\bibitem{c17} J. G. Kreifeldt, ÒAn analysis of surface-detected EMG as an amplitude-modulated noise,Ó presented at the 1989 Int. Conf. Medicine and Biological Engineering, Chicago, IL.
		\bibitem{c18} J. Williams, ÒNarrow-band analyzer (Thesis or Dissertation style),Ó Ph.D. dissertation, Dept. Elect. Eng., Harvard Univ., Cambridge, MA, 1993. 
		\bibitem{c19} N. Kawasaki, ÒParametric study of thermal and chemical nonequilibrium nozzle flow,Ó M.S. thesis, Dept. Electron. Eng., Osaka Univ., Osaka, Japan, 1993.
		\bibitem{c20} J. P. Wilkinson, ÒNonlinear resonant circuit devices (Patent style),Ó U.S. Patent 3 624 12, July 16, 1990. 
		
		
		
		
		
		
	\end{thebibliography}
	
	
	
	
\end{document}