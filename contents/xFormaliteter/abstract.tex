\chapter*{Abstract}
Chronic obstructive pulmonary disease (COPD) is among the leading causes of death worldwide and patients suffering from the disease slowly deteriorate as it progresses. Studies have shown that physical exercise is beneficial to patients suffering from COPD, however some patients are unable or unwilling to leave their homes in order to exercise. This project aims to answer the question of how to motivate COPD patients to exercise and thus improving their general state of health. As an answer to this question a system has been developed which can motivate COPD patients to exercise from home, while still allowing healthcare personnel to monitor their activity. The system is developed mainly using object-oriented programming and the development process is described using unified modeling language. The system is split into three main components; a hardwaremodule for measuring the crankarms revolutions on an exercise bike, a front-end application that displays distance and other information about the training session, and a back-end system with a database and a user interface for healthcare personnel to monitor the patients activiy. 