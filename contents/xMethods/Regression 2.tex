\section{Regression model}
%Preprocessing
%Feature extraction

%regression
This section will include a description of how the regression methods have been executed. As well as data gathering and processing, the regression model will be implemented in MATLAB.

Once the preprocessing and the feature extraction of the EMG data has been done, it is needed to apply regression methods to determine the relationship between the variables under study, since the objective of a regression methods is mapping the response  as a function of the predictions.\cite{hahne2014}

In this particular study, a simple linear regression has been applied. The equation is:

\begin{equation}
	Y_i = \alpha + \beta X_i
\end{equation}

Where $Y_i$ is the response, $\alpha$ is the Y intercept, $\beta$ is the regression coefficient or slope, $X_i$ is the predictor and $i$ is the index \cite{zar2009}.

%capital letters matlab
There has been implemented four different regressors, one for each movement under study. MATLAB is able to generate a linear regression model with one of its build-in mathematical functions. In order to accomplish the regressor the given training data set should include the predictions and the responses.
 The equation which implements the regressor is given in \ref{regressor}. Where Y contains the outputs or the target, this is the response recorded during the training sesion through the GUI. The values were normalize obtaining an optimal value between 0 and 1. X contains the inputs, in this case, the features extracted from each of the eight channel of the Myo armband. The features extracted from the recorded EMG data were the mean absolute value $\left( MAV\right)$ and  the logarithmic variance. %due to its better performance \cite{hahne2014}. 
\begin{equation}
	Y_i=\begin{bmatrix} 
	\ norm_1 \\ 
	\ norm_2\\ 
	\ \vdots\\
	\ norm_8\\
	\end{bmatrix}=
	\alpha +
	\begin{bmatrix} 
	\beta_1 \; \beta_2 \cdots \beta_8\\ 
	\end{bmatrix}
	\cdot 
		\begin{bmatrix} 
	\ X_1 \\ 
	\ X_2\\ 
	\ \vdots\\
	\ X_8
	\label{regressor}
	\end{bmatrix}
\end{equation}

%something else
