\section{Accuracy of regressors}

%head
This section will cover the test used to determine the accuracy of the trained regressors. Both methods are performed on the regressors trained with only EMG data and when IMU data is combined with the EMG data. 

\subsection{Superimposition}
To examine how well the regressors fit the actual data, the output of the regressors build for each feature is superimposed on the actual data. It can then be shown how the regressors perform at which intensities and which movements, and whether other regression methods should be considered to obtain a lower error.  


\subsection{Root Mean Square Error}
To measure the accuracy of the regressors the Root Mean Square Error(RMSE) is calculated. RMSE is a measure to examine how much the regressors disagrees with the actual data. RMSE is a calculation of the standard deviation of the residuals, which is the difference between the estimated values and the actual values. The RMSE is calculated as follows:

\begin{equation}
RMSE = \sqrt{\frac{\sum\limits_{i=1}^N(y_i - \hat{y_i})^2}{N}}
\end{equation}

Where N is the length of the signal, $y_i$ is the $i^{th}$ variable of the actual data and $\hat{y_i}$ is the $i^{th}$ output of the regressor. The RMSE will be done for the regressor of each movement.

To express that the regressors do not over- or under-fit the input data, the RMSE of new test data must be lower than or equal to the data used to train the regressors. The best results for RMSE is as close to zero as possible. 

%It does not necessarily express a great performing model if the RMSE of the training data is high, but a model that performs consistently on new data. However, if the RMSE of the training is reasonably low, and the RMSE of the test data is likely low, the model is said to fit the data well. 