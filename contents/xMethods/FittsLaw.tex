\section{Performance test}

A modified version of Fitts' Law will be used to quantify the performance of the trained regressors. Fitts' Law is a predictive model describing the relation that the time it takes to do a rapid movement to reach a target area, is dependent on the distance to the target area and the size of the target area. The law demonstrates that the information of any human motor tasks, is finite and only limited by the capabilities of the control system. The control exhibit a negative correlation between speed and accuracy. \cite{Kamavuako2014}
Fitts' Law calculates an Index of Difficulty (ID) as given by \eqref{eq:Fitts}

\begin{equation} \label{eq:Fitts}
ID = log_{2} * (\frac{2D}{W})
\end{equation}

Where ID is Index of Difficulty, D is distance to targets and W is width of the target area. However, the system in this study does not provide a reasonable scale for distance and target width. Thus, Fitts' Law cannot be used as usual. Instead only the time is takes a subject to reach the targets and the number of targets reached will be noted to calculate a performance score. The score will be calculated as the average time per reached target.
This performance score indicates that the lower the score the better the performance.

\begin{figure}[H]
	\includegraphics[width=1\textwidth]{figures/Methods/PlacesToGo.png}  %<--but is not needed.
	\caption{The compass plot shows the performed movements and the given intensity of the feature depicted as an arrow originated in origin. The movement performed decides the direction of the arrow and the length is decided by the contraction intensity. The red squares are targets the subject needs to reach. Only one target will be present at a time, but all the targets are depicted in this illustration to show which targets the subjects will be asked to reach.}
	\label{fig:PlacesToGo}
\end{figure}

The test will be done online and consist of reaching 16 targets on time, as shown in \figref{fig:PlacesToGo}. Each target will be present for 30 seconds. If a target is not reached within that time the test will mark the target as missed and move on to the next target. The targets are oriented around origin in two different radii: eight targets close to origin and eight further away. This is done in order to test the proportional control of the regressors. The targets will be fixed on the axes of the compass-plot and in the diagonals in order to test for simultaneous control. \\
The performance of the online test will be compared between the limb positions of the same feature and between the overall performance score of the two features through statistical analysis. After the inclusion of inertial information the performance of the control without IMU data will be compared to the control that includes IMU data. This comparison is additionally performed for the number of targets reached. \\
In the evaluation of the performance scores it will be measured whether the scores obtained from using regressors trained with MAV and LogVar respectively belongs to a normal distribution. This will provide information of which statistical analysis to use. If the data is normal distributed an ANOVA test will be performed, if not, a Friedman's test will be performed.


%To calculate the performance of the human interacting with the control system, the \textit{throughput} (TP) was introduced. IP is calculated by \eqref{eq:TP}.
%
%\begin{equation} \label{eq:TP}
%TP = \frac{ID}{MT}
%\end{equation}
%
%where ID is Index of Difficulty, MT is movement time.
%MT is the time it would take a subject to move a pointer from origin to the target area.
%
%Fitts' Law will be implemented into the test GUI described in \secref{sec:testGUI}, where it will test the control systems ability to correctly convey the test subjects task of reaching points in the compass-plot.
