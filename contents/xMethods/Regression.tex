\section{Regression model}
%Preprocessing
%Feature extraction

%regression
%This section will include a description of how the regression methods have been executed. As well as data gathering and processing, the regression model will be implemented in MATLAB.

Once the preprocessing and the feature extraction of the EMG data has been done, regression will be used as described in \secref{sec:regression}. The implementation follows simple linear regression as shown in \eqref{eq:simpleLinearRegression}. One regressor will be trained for each movement. 


The feature data will be constructed into one matrix containing the extracted features for all four movements at all intensities including the baseline in the three different limb positions arranged into four segments in the matrix. 

This will be set as input to the training of the regressor, while the estimator is a vector containing the mean of all eight channels of the actual data, normalized in relation to the recorded MVC. This provides data for the movement that the regressor will be trained to recognize. The estimator vector is augmented with zeros for the segments of the movements that the regressor is trained not to recognize. This is shown in \eqref{eq:regressionMatrix}

\begin{equation} \label{eq:regressionMatrix}
\begin{bmatrix} 
\ normMove_1 \\ 
\ \vdots\\
\ normMove_n\\
\ 0_o\\
\ \vdots\\
\ 0_p\\
\end{bmatrix}=
\alpha +
\begin{bmatrix}
\ \beta_1\\
\ \vdots\\
\ \beta_8\\
\end{bmatrix} \cdot
\begin{bmatrix} 
\ rightFeat_{1,1} \cdots rightFeat_{1,8} \\ 
\ \vdots \qquad \ddots \qquad \vdots \\
\ rightFeat_{n,1} \cdots rightFeat_{n,8} \\
\ wrongFeat_{o,1} \cdots wrongFeat_{o,8} \\
\ \vdots \qquad \ddots \qquad \vdots \\
\ wrongFeat_{p,1} \cdots wrongFeat_{p,8} \\
\end{bmatrix}
\end{equation}

Where \textit{normMove} is the desired estimator, \textit{rightFeat} is the desired input feature for the desired movement in all limb positions and \textit{wrongFeat} is the features that should not be recognized by the given regressor. The regressor is implemented through the Matlab function \textit{fitlm}, which use the input matrix and estimator vector to calculate the slope and intercept of the regressor.
This procedure is done for each movement, which yields four regressors trained to recognize one movement each. This procedure is done individually for MAV and logarithmic variance features to test the accuracy and performance of the two features. 


When implementing the IMU data three extra columns will be added to the input matrix, because the accelerometer provided a three axis output during recordings. New regressors will be trained using these data also and compared to the regressors trained with only the EMG feature data. 


%
%
% In order to implement this in MatLab the function \textit{fitlm} will be used to build the regressors. 
%
%
%it is needed to apply regression methods to determine the relationship between the variables under study, since the objective of a regression methods is mapping the response as a function of the predictions. \cite{hahne2014}
%
%In this particular study, a simple linear regression has been applied. The equation is:
%
%\begin{equation} %\label{eq:simpleLinearRegression}
%Y_i = \alpha + \beta X_i + \epsilon_i
%%\label{reg}
%\end{equation}
%
%where, $Y$ is the dependent variable or response, $X$ is the independent variable or the predictor, $\beta$ is the regression coefficient or the slope, and $\alpha$ is the Y intercept,  $\epsilon$ is the error and $i$ is the index.\cite{zar2009}.
%
%%capital letters matlab
%There has been implemented four different regressors, one for each movement under study. MATLAB is able to generate a linear regression model with one of its build-in mathematical functions. In order to accomplish the regressor the given training data set should include the predictions and the responses.
%The equation which implements the regressor is given in \ref{regressor}. Where Y contains the outputs or the target, this is the response recorded during the training session through the GUI. The values were normalize obtaining an optimal value between 0 and 1. X contains the inputs, in this case, the features extracted from each of the eight channel of the Myo armband. The features extracted from the recorded EMG data were the mean absolute value $\left( MAV\right)$ and  the logarithmic variance. %due to its better performance \cite{hahne2014}. 
%\begin{equation}
%	Y_i=\begin{bmatrix} 
%	\ norm_1 \\ 
%	\ norm_2\\ 
%	\ \vdots\\
%	\ norm_8\\
%	\end{bmatrix}=
%	\alpha +
%	\begin{bmatrix} 
%	\beta_1 \; \beta_2 \cdots \beta_8\\ 
%	\end{bmatrix}
%	\cdot 
%		\begin{bmatrix} 
%	\ X_1 \\ 
%	\ X_2\\ 
%	\ \vdots\\
%	\ X_8
%	\label{regressor}
%	\end{bmatrix}
%\end{equation}
%
%
