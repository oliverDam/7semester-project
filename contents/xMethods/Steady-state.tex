\section{Steady-state}
%ideas

The EMG signal is a non-linear and non-stationary stochastic process. It is the result of the addition of different motor unit action potential trains  (MUAPTs). Even though, two main states have been identified, the transient state and the steady state. The transient state is related with the beginning phase of the muscle contraction, while the steady state is consider the myoelectrical signal during the stable phase of the muscle contraction when the final position is achieve. \cite{mobarak2014}

Although the steady state contains a short temporal structure of the patterns involved in the contraction of the muscle \cite{mobarak}, there have been developed diffierent studies where is possible to achieve online continous control using steady-state EMG signals. A study by Englehart et al. \cite{} demonstrated that steady-state data clasified precisely than transient data. This could be due to the fact of largest amount of meaningful data in this muscle contraction phase \cite{mobarak}  %However there are still limitations due the fact that the classifier can not deal with the transient EMG signals. Some clinical applications that combine both data have shown an increase in the recognition system.  

For this particular study the EMG signals were addressed in steady-state which coincide with the plateu phase in the trapeze of the GUI that has been employ fot the data acquisition. This decision was made based on the different studies which concluded that is possible to achieve simultaneous and proportional control of hand prosthesis employing the steady state information of the EMG signals.

