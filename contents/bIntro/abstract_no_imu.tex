%abstract
\section{Abstract}

<<<<<<< HEAD
<<<<<<< HEAD
\textbf{Background:} Electromyography (EMG) is widely used as input to the control scheme of myoelectric prosthetics. However, EMG signals change with limb position and thus lowers the accuracy in pattern recognition \cite{Fougner2010}.% and very few studies has assessed this problem.
Most studies have formerly utilized classification for pattern recognition when changing limb position.  
%Inclusion of the use of Inertial Measurement Units (IMU) has proved to raise the accuracy in pattern recognition methods.
However, pattern recognition provides only control of one degree of freedom (DoF) at a time, and are computational costly. This study will investigate the usability of linear regression to classify movements to make a control scheme for myoelectric prosthetics, which has proven to yield robust control\cite{hahne2014}\. The aim is to test and compare the performance of two different features (Mean Absolute Value and Logarithmic variance) when used to recognise movements of four movements at the wrist in three different limb positions. 
=======
\textbf{Background:} Electromyography (EMG) is widely used as input to control scheme of myoelectric prosthetics. However, EMG signals change with limb position and thus lowers the accuracy in classification.% and very few studies has assessed this problem. 
Inclusion of the use of Inertial Measurement Units (IMU) has proved to raise the accuracy in pattern recognition methods. However, pattern recognition methods provides only control of one degree of freedom (DoF) at a time, and are computational costly. This study propose to use the combination of EMG recordings and accelerometer data in a linear regression model to overcome the slower reaction time of pattern recognition systems and to enable a simultaneous and proportional control scheme. 
>>>>>>> b5392b1dd3e5f6c5d81abdf28134a674b70fe3df


<<<<<<< HEAD

\textbf{Methods:} In this study recordings from eight able-bodied subjects has been collected. The data is evaluated through principal component analysis (PCA) and trained with a linear regression model of which the Root Mean Square Error is calculated to evaluate the offline performance. One regressor is trained for each wrist movement for each test subject; four with each feature. The regressors are tested online in a visual environment measuring time to complete a target-reaching task of sixteen targets. The performance of the online test is compared between the different limb positions of the same feature and all limb positions of the two features through statistical analysis.
%the IMU data and not using IMU data to determine the effect of including IMU data. 
=======
\textbf{Methods:} In this study recordings from four able-bodied subjects has been collected, performing four hand movements at the wrist in three different limb positions. The data is evaluated through principal component analysis (PCA) and processed/trained with a linear model to classify the hand movements. Eight regressors are trained for each test subject; four with and without using IMU data. The regressors are tested in a real-time visual environment on PC measuring time to complete a target-reaching task of eight targets. The performance of the regressors are compared between using the IMU data and not using IMU data to determine the effect of including IMU data. 
>>>>>>> b5392b1dd3e5f6c5d81abdf28134a674b70fe3df
=======
\textbf{Background:} Electromyography (EMG) is widely used as input to control scheme of myoelectric prosthetics. However, EMG signals change with limb position and thus lowers the accuracy in classification.% and very few studies has assessed this problem. 
Inclusion of the use of Inertial Measurement Units (IMU) has proved to raise the accuracy in pattern recognition methods. However, pattern recognition methods provides only control of one degree of freedom (DoF) at a time, and are computational costly. This study propose to use the combination of EMG recordings and accelerometer data in a linear regression model to overcome the slower reaction time of pattern recognition systems and to enable a simultaneous and proportional control scheme. 


\textbf{Methods:} In this study recordings from four able-bodied subjects has been collected, performing four hand movements at the wrist in three different limb positions. The data is evaluated through principal component analysis (PCA) and processed/trained with a linear model to classify the hand movements. Eight regressors are trained for each test subject; four with and without using IMU data. The regressors are tested in a real-time visual environment on PC measuring time to complete a target-reaching task of eight targets. The performance of the regressors are compared between using the IMU data and not using IMU data to determine the effect of including IMU data. 
>>>>>>> b5392b1dd3e5f6c5d81abdf28134a674b70fe3df


%modified Fitts' Law 


\textbf{Results:} 


\textbf{Conclusion:} 


\textbf{Keywords:} surface electromyography, simultaneous and proportional myoelectric control, lower arm prosthetics, regression, machine learning %inertial measurement unit, hand motion classification,
