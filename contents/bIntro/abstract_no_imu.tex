%abstract
\section{Abstract}

\textbf{Background:} Electromyography (EMG) is widely used as input to control scheme of myoelectric prosthetics. However, EMG signals change with limb position and thus lowers the accuracy in classification.% and very few studies has assessed this problem. 
Inclusion of the use of Inertial Measurement Units (IMU) has proved to raise the accuracy in pattern recognition methods. However, pattern recognition methods provides only control of one degree of freedom (DoF) at a time, and are computational costly. This study propose to use the combination of EMG recordings and accelerometer data in a linear regression model to overcome the slower reaction time of pattern recognition systems and to enable a simultaneous and proportional control scheme. 


\textbf{Methods:} In this study recordings from four able-bodied subjects has been collected, performing four hand movements at the wrist in three different limb positions. The data is evaluated through principal component analysis (PCA) and processed/trained with a linear model to classify the hand movements. Eight regressors are trained for each test subject; four with and without using IMU data. The regressors are tested in a real-time visual environment on PC measuring time to complete a target-reaching task of eight targets. The performance of the regressors are compared between using the IMU data and not using IMU data to determine the effect of including IMU data. 


%modified Fitts' Law 


\textbf{Results:} 


\textbf{Conclusion:} 


\textbf{Keywords:} surface electromyography, inertial measurement unit, simultaneous and proportional myoelectric control, regression, hand motion classification, hand prosthetic
