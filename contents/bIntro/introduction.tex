%Introduction

%Why prostheses are important?


%What is an EMG-prosthesis?


%What has been done in the EMG-prosthesis area?
In recent years the development of EMG controlled prosthetics have advanced considerably, due to an increased interest in the area along with a higher demand for better prosthetics and more precise control. \cite{Fougner2012} In the early years most EMG prosthetics functioned by only controlling one DOF by \textit{on-off control}, mostly by linking antagonistic muscles to one DOF. This along with \textit{mode switching} provided users a way to control more than one DOF, but never simultaneously. However, as demands would rise, more complex methods was introduced to the EMG scene, and proportional control was introduced with pattern recognition methods. This effectively enabled simultaneous control of more than one DOF, but gave rise to new problems; a wider range of control would give less accurate movements, and training the pattern recognition methods proved difficult, as the training could overfit, causing extended use of the prosthetics to degrade in performance. \cite{Ison2016} More advanced prosthetics have also been developed making it possible to control several more DOF, especially for individual finger movements. However, no EMG-based control scheme has been able extract an adequate amount of information to effectively control these advanced prosthetics. \cite{hahne2014}


A study by Fougner et al. \cite{Fougner2011} have addressed the problem that most studies test their algorithm/method on only one position.
%Which issues are there in the EMG-prosthesis area?(decreasing quality of control of hand gestures when the arm is placed in different positions)
This proves a problem when it have been shown that muscles create muscle-synergies to perform movements, and so a change can be seen in recorded EMG signals from muscles when the arm is positioned in different positions. \cite{avella2006} \cite{DeRugy2013} \cite{Fougner2011} 
%What would be novel to add to this area?(adding IMU’s to a regressor, since it has been done with a classifier)
In order to overcome this problem Fougner et al. \cite{Fougner2011} has suggested to combine recording of EMG signals with data from an accelerometer to provide arm position data, would be beneficial in increasing the accuracy of EMG controlled prosthetics. Fougner et al. used linear discriminant analysis with four time domain features (mean absolute value, zero crossing, number of turns, waveform length) to analyse the EMG signals. They used the acquired position data to form feature vectors to represent different arm positions. They then classified the data in four different training schemes, with results showing improvement in classification, reducing average error from 18\% to 5\%. \cite{Fougner2011}
A novel approach to improve on these findings would be to include data from inertial measurement units (IMU) to the training of the regressor. 

%Hypothesis
It is possible to do proportional and simultaneous control of two DOFs in a lower-arm prosthesis, while having the arm in different positions, using simple/multiple linear regression on recorded surface EMG signals and inertial measurement units. % to something...?

%It is possible to control multiple DOF’s of a JACO robotic arm (Kinova) using quaternions, while also being able to control two DOF’s (wrist-rotations and open/close of hand) of the end-effector, which is a three-fingered hand, using multiple regression processing of EMG signals measured from the forearm using a MYOBAND (Thalmic Labs)

