%Introduction

%Why prostheses are important?
Upper limb prostheses have the purpose of fulfilling the users demand, which consists of cosmetic and functional support. The utmost wish for the consumer is to regain full appearance and function of the missing biological upper limb. The functionality is the most challenging aspects to fulfil. Two types of functional prostheses exist: body-powered and electrical, where the electrical has the highest functionality, and therefore ideally has a higher similarity to a biological upper limb. The most common electric prosthesis is the myoelectric prosthesis, where EMG signals are used as the control signal. \cite{jiang2012}. 
%What is an EMG-prosthesis?
%The performance of myoelctric prosthesis is based on the surface EMG signals adquisition from the muscles for further processing in order to activate different functions in the prosthesis

%What has been done in the EMG-prosthesis area?
In recent years the development of EMG controlled prosthetics have advanced considerably, due to an increased interest in the area along with a higher demand for better prosthetics and more precise control. \cite{Fougner2012} In the early years most EMG prosthetics functioned by only controlling one degree of freedom (DOF) by \textit{on-off control}, mostly by linking antagonistic muscles to one DOF. This kind of prostheses changes between states due to a switching impulse which cause a state machine to shift its present state. Usually a strong and fast muscle contraction from the users are employed to generate the switching signals. \cite{amsuess2014}
This type of control provided users a way to control more than one DOF, but never simultaneously. The switch-control functioned on a cycle, so users would have to go through all the movements of the prosthesis to find the one they wanted to perform. However, as demands would rise, more complex methods was introduced to the EMG prosthetics scene. Classification methods effectively enabled users to use DOF's more freely because the switching was now replaced by direct recognition of different muscle contractions linked to specific prosthetic movements. This also effectively enabled proportional control of movements, but gave rise to new problems; a wider range of control would give less accurate movements, and training the classifiers proved difficult, as the training could over-fit, causing extended use of the prosthetics to degrade in performance. \cite{Ison2016}

%More advanced prosthetics have also been developed making it possible to control several more DOF, especially for individual finger movements. However, no EMG-based control scheme has been able extract an adequate amount of information to effectively control these advanced prosthetics.
Introducing regression as a new mapping method in myoelectric prosthetics provided a way to enable both simultaneous and proportional control of multiple DOF's. This is because regression is able to provide a continuous value for each DOF based on the recorded EMG signal, while a classifier only decides upon a certain class.% while classification function on a discrete approximation of the continuous parameter space of a muscle contraction. \cite{hahne2014, jiang2010} 
This means that classification can only translate a recorded EMG signal to one movement of the prosthetic at a time. It can do so proportionally but the handling still lacks natural control, because movements by able-bodied individuals very rarely only happen in one DOF at a time. Regression methods constantly provide a value, and since several regressors can be used at a time, several values can be used in the recognition of movements. This is what enables regression methods to perform simultaneous and proportionally. 

%In pattern recognition methods the patterns extracted from muscle contraction are used to obtain a discrete approximation of the continuous parameter space, however this leads to a lack of natural control. Simultaneous and proportional control is not possible due to the fact that one class is selected in each decision and proportional control is implemented after the classification phase \cite{jiang2010}. On the other hand regression methods are , in this way proportional and simultaneous control is achieved \cite{hahne2014}. }  

Applying regression as a mapping method in proportional and simultaneous control of multiple DOF's has been shown to perform well in recognition of movements and doing so with a low computation time. \cite{hahne2014} However, very few studies have tested the regressor performance in daily life tasks outside the clinical training environment. \cite{jiang2012} A study by Fougner et al. \cite{Fougner2011} has addressed the problem that most studies test their method on only one limb position. This means that the actual performance of regression methods has not yet been properly addressed when recognizing movements where the arm is in positions that is normally a part of daily life tasks. 

%Which issues are there in the EMG-prosthesis area?(decreasing quality of control of hand gestures when the arm is placed in different positions)
When recording EMG signals it has been shown that some muscles are activated based on joint angles, even though the muscles are not involved in the movement of that joint \cite{Fougner2011}. This provides a problem, but can be explained by muscle-synergies, which have been shown to exist between muscles \cite{DeRugy2013}. These muscle-synergies are created by the Central Nervous System (CNS) and coordinated into activation of different muscles at varying times. This enables the CNS to control the muscle-synergies instead of controlling each muscle individually, to perform movements \cite{jiang2009}. This means that muscles in the lower arm can be activated when muscles in the upper arm are activated, enough so that it would be recordable on EMG recordings, and enough to alter recognition of movements when the arm is active in limb positions other than the one tested in a clinical environment. 
%and so a change can be seen in recorded EMG signals from muscles when the arm is positioned in different positions. \cite{Fougner2011, avella2006, DeRugy2013}

%What would be novel to add to this area?(adding IMU’s to a regressor, since it has been done with a classifier)
In order to overcome the problem of muscles activating, when movements other than those they are involved in are active, Fougner et al. \cite{Fougner2011} has suggested to combine recording of EMG signals with inertial information(IMU) to provide limb position data. This could be beneficial in increasing the accuracy of EMG controlled prosthetics. 
Even though the combination of EMG and IMU data has been proposed as a valid way to improve the performance and accuracy of EMG based prosthetics, it has only been investigated in few studies. \cite{Roy2010, Imtiaz2014, jiang2012}
%Fougner et al. used linear discriminant analysis with four time domain features (mean absolute value, zero crossing, number of turns, waveform length) to analyse the EMG signals. They used the acquired position data to form feature vectors to represent different arm positions. They then classified the data in four different training schemes, with results showing improvement in classification, reducing average error from 18\% to 5\%. \cite{Fougner2011} 
%Adding inertial measurement units (IMU) to the mapping of hand gestures in different limb position has to our knowledge only been done with classification methods. \textbf{SOURCES} 
To the authors knowledge the use of the combination of EMG recordings and IMU data has only been done with classification methods. A novel approach to further investigate the usability of combining EMG and IMU is to build a regression based control scheme for myoelectric prosthetics. This would enable both proportional and simultaneous control of several DOF's, where the inclusion of IMU data should provide more information on limb position to counter the effect of muscle-synergies. 
This leads to the following hypothesis:
\begin{center}
	 It is possible to do proportional and simultaneous control of two DOFs in a lower-arm prosthesis, while having the arm in different positions, using linear regression on recorded surface EMG signals and inertial measurement data.
\end{center}



%the training of a regressor, as this to 

%improve upon the findings of improvement of performance and accuracy of control when combining EMG and IMU, 


 %findings would be to include data from IMU, to the recordings of EMG, to build a regression based control scheme for prosthetics. This the training of a regressor, as this to 
%Hypothesis
 % to something...?

%It is possible to control multiple DOF’s of a JACO robotic arm (Kinova) using quaternions, while also being able to control two DOF’s (wrist-rotations and open/close of hand) of the end-effector, which is a three-fingered hand, using multiple regression processing of EMG signals measured from the forearm using a MYOBAND (Thalmic Labs)

