%Introduction

%Why prostheses are important?
Upper limb prostheses have the purpose of fulfilling the costumers demand, which consists of cosmetic and functional support. The utmost wish for the consumer is to regain full appearance and function of the missing biological upper limb. The functionality is the most challenging aspects to fulfil. Two types of functional prostheses exist: body-powered and electrical, where the electrical has the highest functionality, and therefore ideally has a higher similarity to a biological upper limb. The most common electric prosthesis is the myoelectric prosthesis, where EMG signals are used as the control signal. \cite{jiang2012}. 
%What is an EMG-prosthesis?
%The performance of myoelctric prosthesis is based on the surface EMG signals adquisition from the muscles for further processing in order to activate different functions in the prosthesis

%What has been done in the EMG-prosthesis area?
In recent years the development of EMG controlled prosthetics have advanced considerably, due to an increased interest in the area along with a higher demand for better prosthetics and more precise control. \cite{Fougner2012} In the early years most EMG prosthetics functioned by only controlling one DOF by \textit{on-off control}, mostly by linking antagonistic muscles to one DOF. This along with \textit{mode switching} provided users a way to control more than one DOF, but never simultaneously. However, as demands would rise, more complex methods was introduced to the EMG scene, and proportional control was introduced with pattern recognition methods. This effectively enabled simultaneous control of more than one DOF, but gave rise to new problems; a wider range of control would give less accurate movements, and training the pattern recognition methods proved difficult, as the training could overfit, causing extended use of the prosthetics to degrade in performance. \cite{Ison2016} 
%More advanced prosthetics have also been developed making it possible to control several more DOF, especially for individual finger movements. However, no EMG-based control scheme has been able extract an adequate amount of information to effectively control these advanced prosthetics.
Applying regression as a new mapping method in proportional and simultaneous control of multiple DOF's has been shown to perform well and doing this with a low computation time. \cite{hahne2014} In spite of the proportional and simultaneous control systems performing decently, a problem still occurs when the prostheses are to mimic daily life tasks outside the clinical training environment. \cite{jiang2012}
%a hole here need to be linked to the next text
%the next text:
A study by Fougner et al. \cite{Fougner2011} has addressed the problem that most studies test their method on only one limb position.
%Which issues are there in the EMG-prosthesis area?(decreasing quality of control of hand gestures when the arm is placed in different positions)
This proves a problem when it have been shown that muscles create muscle-synergies to perform movements, and so a change can be seen in recorded EMG signals from muscles when the arm is positioned in different positions. \cite{Fougner2011, avella2006, DeRugy2013} 
%What would be novel to add to this area?(adding IMU’s to a regressor, since it has been done with a classifier)
In order to overcome this problem Fougner et al. \cite{Fougner2011} has suggested to combine recording of EMG signals with data from an accelerometer to provide upper limb position data, would be beneficial in increasing the accuracy of EMG controlled prosthetics. Even though the combination of EMG and IMUs has been proposed as a valid way to further improve upon EMG based prosthetics, and has in other research areas been acknowledged as a mean to acquire higher level of classification accuracy, is has only rarely been reported in studies. \cite{Roy2010, Imtiaz2014, jiang2012}
%Fougner et al. used linear discriminant analysis with four time domain features (mean absolute value, zero crossing, number of turns, waveform length) to analyse the EMG signals. They used the acquired position data to form feature vectors to represent different arm positions. They then classified the data in four different training schemes, with results showing improvement in classification, reducing average error from 18\% to 5\%. \cite{Fougner2011} 
%Adding inertial measurement units (IMU) to the mapping of hand gestures in different limb position has to our knowledge only been done with classification methods. \textbf{SOURCES} 
A novel approach to improve upon these findings would be to include data from IMU, to the recordings of EMG, to the training of a regressor, as this to the authors knowledge only have been done with classification methods. 
%Hypothesis
It is possible to do proportional and simultaneous control of two DOFs in a lower-arm prosthesis, while having the arm in different positions, using simple/multiple linear regression on recorded surface EMG signals and inertial measurement units. % to something...?

%It is possible to control multiple DOF’s of a JACO robotic arm (Kinova) using quaternions, while also being able to control two DOF’s (wrist-rotations and open/close of hand) of the end-effector, which is a three-fingered hand, using multiple regression processing of EMG signals measured from the forearm using a MYOBAND (Thalmic Labs)

