\section{Regression methods}

%Linear regression:
	%linear regression (how well the line fits the data can be decided with goodnes of fit or least squares)
	%multiple linear regression
	%principal components regression
	%bayesian linear regression
%non-linear regression:
	%kernel ridge regression
	%gaussian regression 
	

%head
Regression methods are widely used is statistics as a method to determine relationship between variables, and to extract a relation that can be used as a way to predict future developments or tendencies in a given data set. It is also a commonly used method to evaluate EMG signals to determine different parameters. There exist many regression methods, but overall to classes of methods can be defined; linear and non-linear, some of which will be covered in this section. 

In the linear class the most basic form of regression is the linear regression, which is a fit of a straight line to best fit several data points. This regression method is widely used in studies where it is used to describe a simple relationship between a dependent and independent factor.

A variant of the linear regression is the multiple linear regression, which can be used in cases where one dependent variable and several independent are to be described and a generalization of the relationship between the variables wish to be found. 

Principal component regression is based on the principal component analysis, where a very large dataset can be compressed into a few of the most important components to further be analyzed upon. This means that in a dataset of many samples, the ones which are the most responsible or most correlated in a meaningful relationship between the dependent and independent variables, can be isolated to best describe which factors have a result on the relationship.







%tail
