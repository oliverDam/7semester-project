\section{What do we measure with EMG?}

The electric potential detected with an electromyography is an action potential causing the muscle to contract. Certain mechanisms are involved for this to happen. The motor unit of the muscle needs to be activated alongside with its associated alpha motor system, which is the lower motor neuron, its axon, and the muscle fibers the motor unit innervates. The muscle fiber is an excitable cell with a resting potential of between -90mV and -70mV. A threshold of approximately -55mV needs to be reached for an action potential to be generated. The sarcolemma, the membrane covering the muscle fibers, has sodium and potassium ion channels that maintains the resting potential, depolarize the muscle fiber if the threshold is exceeded or repolarize the muscle fiber.

The lower motor axon is branching out so that it can attach to the muscle fiber at the motor end-plate and create neuromuscular synapses. The action potential traveling down the axon reaches the synapses and releases ACh. ACh raises the permeability of the cell membrane where sodium ions influx and causes the membrane to depolarize. Calcium ions are released and binds with troponin and exposes the active sites on the thin filaments which allows the muscle to contract. The action potential travels along the whole muscle fiber through t-tubuluses. This happens in both directions from the motor end-plate to the tendentious attachment. When the peak of the depolarization of about 30mV is reached a rapid efflux of potassium ions causes the muscle fiber to repolarize and reach its resting potential again.

Depending on the force that needs to be applied for a given task more or less motor units are activated and therefore more or less muscle fibers are contracted. The bigger the force the more motor units are activated. Furthermore, the number muscle fibers per motor unit varies between muscles in the human anatomy. The finer the movement the higher the innervation, e.g. the extraocular muscle has the highest innervation of 3:1 and the gastrocnemius muscles has one of 2000:1.

(Something about how the innervation is in certain muscles of the forearm)