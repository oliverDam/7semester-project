\section{JACO$^2$ robotic arm}

In this section a briefly description of the JACO$^2$ robotic arm will be given. It is a 6 DOF robotic arm  with a three fingered hand developed by Kinova Robotics. Its lightweight $\left( 4.4 kg\right)$  makes this machine specially indicated for use in assistive and collaborative applications. It is design to help people with upper body disabilities in order to gain more autonomy in ordinary daily tasks.\\

\begin{figure}[H]                    
\includegraphics[width=.3\textwidth]{figures/Jaco/roboticarm}  %<--but is not needed.
\caption{. \cite{}}
\label{fig:roboticarm}  %<--give the figure a label, so you can reference!
\end{figure}
The JACO$^2$  is a serial manipulator, which means that this kind of robotic arms are designed as a series of links connected by motor-actuated joints that extend from a base to an end-effector. Any movement in a joint affects all the following joints and links in the chain. The arm can be controlled with the help of a joystick, it can be programmed in C++, using an SDK provided by the manufacturer.\\


The control of the JACO$^2$ arm can be Angular and Cartesian. In Angular control each actuator moves separately. Cartesian robots called Gantry robots as well, are mechatronic devices which make linear movements in three axes, perpendicularly oriented to each other. It allows eight movements:
\begin{itemize}
\item Three translations
\item Three rotations of the wrist
\item Two movements of the fingers $\left( open/close\right) $
\end{itemize}

