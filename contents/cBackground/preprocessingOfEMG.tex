\section{Preprocessing of EMG}

In order to achieve a higher signal to noise ratio (SNR) it is common practice to implement some preprocessing methods, including input impedance, differential amplification and filtering. The raw EMG signals has to be preprocessed due to them being sensible to noise elements from the surroundings, since the range of the signal is in the order of millivolts to microvolts. Input impedance is determined by a simple rule in order to avoid defeating the common mode rejection of the EMG amplifier. The rule states that the input impedance of the EMG amplifier has to be between $10$ and $100$ times higher than the impedance of the skin-electrode interface.\cite{cram2012}

Differential amplification is used in EMG in order to amplify the original signal and remove common signals from two or more electrodes, in order to avoid common noise from more electrodes in the amplified signal. The amplifier must have a build in gain as well \fxnote{check upon what the preprocessing properties are for the Myo armband}, which determines the final strength of the signal, and both of these features are implemented in order to avoid the SNR. Basic filtering should be implemented in order to avoid electrical noise (50Hz). This filter would be implemented as a notch filter, in order to reject the electrical noise and achieve a higher SNR. Furthermore the filtering should include a bandpass filter with a bandwith chosen depending on where the EMG is performed. This is done in order to make sure the final signal doesn't contain irrelevant high and low frequencies.\cite{cram2012}

\textbf{add stuff about the bandwidth of the EMG signal (5-500Hz), the implementation of a high-pass filter to avoid low frequencies from e.g. movement artefacts, not implement a band-pass filter because the Myo armband only detect between between 0-100Hz, and therefore the highest frequencies we will ever detect lies within the EMG bandwidth.}