I%ntroduction
%Why prostheses are important?
%What is an EMG-prosthesis?
%What has been done in the EMG-prosthesis area?
%Which issues are there in the EMG-prosthesis area?(decreasing quality of control of hand gestures when the arm is placed in different positions)
%What would be novel to add to this area?(adding IMU’s to a regressor, since it has been done with a classifier)
%Hypothesis

Classic upper-limb EMG prostheses use two electrode pairs, placed on antagonistic muscles, to control one degree of freedom (DOF) [1]. Multifunctional prostheses may pro- vide several more DOF, for example: to control individual fingers [2], or units for wrist (flexion–extension, supination– pronation) and hand (open–keygrip–opposition-grip)
If a prosthesis provides more than one DOF, switching (e.g. co-contraction of antagonist muscle pairs) of control between DOFs is re- quired [3]. Increasing the number of electrodes to gain more signals is effective only to a limited degree, since muscles work mostly synergistic, i.e. their activity is correlated in most tasks, and because the surface EMG detection is limited by relative large crosstalk between nearby electrodes [4].