\section{Electromyography acquisition}


%%%%%%%% !!!!!!!!! SECTION MOVED TO THE END OF MYOBAND SECTION !!!!!!!!! %%%%%%%%%

The following section will contain an explanation of the main component of acquiring EMG signals using surface electrodes.
%, including different electrode designs and the most common preprocessing methods used in EMG acquisition.

When acquiring EMG signals the electrodes act as a transducer by converting the differences in ion distribution on the skin surface caused by ion exchange under muscle activity, into an electric current. Surface electrodes used to aquire EMG signals comes both with and without gel covered surfaces, where the the Myo armband employs dry electrodes. Using dry electrodes will often be more practical in use, while the gel covered electrodes will aquire more exact readings of the signals. \cite{lee2008, cram2012}

The most commonly used electrodes for EMG are made of disposable silver-impregnated plastic, and in order to keep the electric potential on the skin surface stable and reduce impedance between the surfaces, they are often covered in a silver chloride gel. Using dry electrodes will result in  a higher surface impedance, which means that the signal contains more noise compared to a gel covered electrode. However, when using dry electrodes the skin will itself provide a “gel” by sweating which will increase readings and decrease the impedance. \cite{cram2012}

