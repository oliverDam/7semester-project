\section{"Conclusion" of the background}

This project will focus on the mapping of different hand gestures while having the arm in different positions. This mapping relies on that the generated EMG from the different hand gestures are differentiable. For a prosthetic user a good performing prosthesis must perform hand gestures as well in an elevated limb position as in a seated position to be able to support the user in daily tasks, e.g. taking a cup from a cupboard and pouring water into the cup. However, changes in the EMG occurs when performing the same hand gestures in different limb positions \cite{Fougner2011, avella2006}. These signal alternations can occur for different reasons. Changing limb position can make muscles move under the skin, relative to the placement of the EMG electrodes, resolving in change of the signal source. Muscle contractions in them selves can also make changes to the recorded EMG due to change in the microscopic structure of the muscles caused by overlap of thick and thin filaments. \cite{martini}  
%can lengthen the muscles and result in a change in the signal source relative to the electrode from which the EMG signal is obtained, and even the lengthening of the muscles itself due to changing limb position will alter the EMG activity caused by a degree of overlap of the thick and thin filaments. 
Other findings has shown that the activity of certain muscles' is depending on angles of joints besides those primarily actuating the contraction of these muscles. \cite{Fougner2011} Thus, this limb position effect must be seen as an important aspect to take into consideration in the mapping of hand gestures to control a prosthesis for the user to receive a good performing support device. 

%review papers which have wokred with the problem of detecting different EMG signals with that arm in different positions. 
%also point out that the problem is known and have been addressed in several studies. 
	% use IMU: imtiaz2014, roy2010, fougner2011
  	% propose use of IMU: jiang2012, fougner2010, 